\chapter{Opis procesów biznesowych}
\label{Chapter2}

Niniejszy rozdział przedstawia otoczenie systemu LinkedInGrads. Wyszczególnieni zostali aktorzy: użytkownicy oraz zewnętrzne systemy. Zaprezentowano obiekty biznesowe będące rzeczywistością, w jakiej porusza się użytkownik systemu. Każdy obiekt opatrzono krótkim opisem oraz wykazem atrybutów. Ostatni podrozdział prezentuje biznesowe przypadki użycia.

\section{Aktorzy i zewnętrzne systemy}
\label{Chapter21}

\begin{itemize}
\item Administrator - osoba odpowiedzialna za aktualizację danych w systemie: dodawanie nowych absolwentów, znajdowanie adresów profili absolwentów.
\item Pracownik dziekanatu - główny użytkownik systemu, grupuje pojęcia dotyczące absolwentów, generuje raporty.
\item eLogin - zewnętrzny system Politechniki Poznańskiej służący do uwierzytelniania użytkowników.
\item LinkedIn - sieć społecznościowa zrzeszająca specjalistów, służąca nawiązywaniu kontaktów i rozwojowi kariery.
\end{itemize}

\section{Obiekty biznesowe}
\label{Chapter22}

\subsection{Raport}

Raport jest obiektem biznesowym reprezentującym wycinek danych przetworzonych przez system. Dodatkowo każdy z raportów uwzględnia nagłówek zawierający podstawowe informacje na temat raportu oraz zestawienie stosunku ilości danych pochodzących z różnych źródeł. Raport może również zawierać uzasadnienie prezentowanych danych prezentując dodatkowo listę absolwentów wraz z wartością użytych do wygenerowania raportu.

\begin{itemize}
\item Analiza pracodawców - prezentuje ilość absolwentów zatrudnionych w danych firmach
\item Analiza zamieszkania - prezentuje miejsca zamieszkania absolwentów
\item Analiza umiejętności - prezentuje umiejętności nabyte przez studentów
\item Analiza zatrudnienia - prezentuje stosunek osób zatrudnionych do bezrobotnych
\item Analiza stanowisk - prezentuje ilość studentów na danych stanowiskach, które mogą zostać zgrupowane do bardziej uniwersalnego nazewnictwa. Dodatkowo uwzględnia historyczne stanowiska absolwentów.
\end{itemize}

Atrybuty:

\begin{itemize}
\item Typ raportu,
\item Rok ukończenia studiów,
\item Uzasadnienie danych.
\end{itemize}

\subsection{Grupa}
\label{subsection-grupa}

Grupa jest obiektem biznesowym reprezentującym alias dla nazewnictwa użytego w danych pobranych z zewnętrznych źródeł. Obiekt ten wykorzystywany jest wewnątrz raportu w celu unifikacji nazewnictwa użytego w zewnętrznych źródłach.

Atrybuty:

\begin{itemize}
\item zgrupowana nazwa,
\item nazwa pierwotna.
\end{itemize}

\pagebreak
\section{Biznesowe przypadki użycia}
\label{Chapter23}

\subsection{Generowanie raportu o absolwentach}

\bucsection{BUC1: Generowanie raportu o absolwentach}{Pracownik dziekanatu}
{Istnieją dane użytkowników pobrane z zewnętrznych źródeł i wzorce raportów}{Raport}
{\ucactions{
\ucaction{1. Pracownik wybiera typ raportu.}
\ucaction{2. Pracownik wybiera filtry raportu.}
\ucaction{3. Pracownik analizuje otrzymany raport.}
}}
{}

\subsection{Łączenie pojęć w grupy}

\bucsection{BUC2: Łączenie pojęć w grupy}{Pracownik dziekanatu}
{Istnieją dane użytkowników pobrane z zewnętrznych źródeł}{Grupa}
{\ucactions{
\ucaction{1. Pracownik tworzy nową grupę.}
\ucaction{2. Pracownik przypisuje pojęcia do grupy.}
}}
{}