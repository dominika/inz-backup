\chapter{Opis procesów biznesowych}
\label{Chapter2}

\section{Aktorzy}
\label{Chapter21}

Wymienienie aktorów zdefiniowanych dla systemu (np. jako wyliczenia).

\section{Obiekty biznesowe}
\label{Chapter22}

\subsection{Obiekt 1}

Opis obiektu pierwszego, jego zastosowania. Poniżej wymieniamy jego atrybuty (jeśli istnieją):

\begin{itemize}
\item Atrybut 1
\item Atrybut 2
\item ...
\end{itemize}

\subsection{Obiekt 2}

Ponownie opis, tym razem dla obiektu drugiego. Oczywiście, inne obiekty opisujemy tak samo:

\begin{itemize}
\item Atrybut 1
\item Atrybut 2
\item ...
\end{itemize}

\pagebreak
\section{Biznesowe przypadki użycia}
\label{Chapter23}

Poniżej przedstawiacie biznesowe przypadki użycia. Proponuję tutaj wykorzystać szablon, który również jest przerobionym szablonem LaTeXowym i dostosowanym do naszych potrzeb, specjalnie dla opisywania przypadków użycia.

\subsection{ID01: Nazwa przypadku użycia}

\ucsection{ID01: Nazwa przypadku użycia}{Aktor 1, Aktor 2}
{}{}
\ucactions{
\ucaction{1. Pierwszy punkt głównego scenariusza}
\ucaction{2. Drugi punkt głównego scenariusza}
\ucaction{3. Trzeci punkt głównego scenariusza}
}}
{\ucextensions{
\ucaction{3.A Opis sytuacji wyjątkowej 1}
\ucaction{3.A.1 Pierwszy krok sytuacji wyjątkowej 1}
\ucaction{3.B Opis sytuacji wyjątkowej 2}
\ucaction{3.B.1 Pierwszy krok sytuacji wyjątkowej 2}
\ucaction{3.B.2 Drugi krok sytuacji wyjątkowej 2}
}}
{}

\subsection{ID02: Przykład przypadku użycia bez wyjątków}

\ucsection{ID02: Przykład przypadku użycia bez wyjątków}{Aktor 1}
{}
{}{\ucactions{
\ucaction{1. Pierwszy punkt głównego scenariusza}
\ucaction{2. Drugi punkt głównego scenariusza}
\ucaction{3. Trzeci punkt głównego scenariusza}
}}
{}

\section{Opis procesów biznesowych z wykorzystaniem notacji graficznej}
\label{Chapter24}

Ten podrozdział jest opcjonalny (a może nawet można to przedstawić jako podsekcję poprzedniego podrozdziału). Tutaj możemy pochwalić się diagramem (np.~BPMN) dla naszych procesów biznesowych. Oczywiście, nie tylko w formie graficznej -- jeśli macie inną formę, to też się nadaje (o ile jest ona przyzwoita).