\chapter{Zakończenie}
\label{Chapter9}

\section{Podsumowanie}
\label{Chapter91}

Budowa systemu wykorzystującego serwis LinkedIn jako źródło danych powinno pozwolić nie tylko na dotarcie do informacji o większej liczbie absolwentów, ale przede wszystkim na automatyzację procesu monitorowania, a zatem obniżenie jego kosztu czasowego i finansowego.

Realizacja systemu LinkedInGrads powiodła się w wyznaczonym zakresie umożliwiając dotarcie Politechnice Poznańskiej do danych o swoich absolwentach. Długofalowe skutki wdrożenia aplikacji z pewnością zaistnieją, a sam system dzięki swojej modularności pozwala na dalszą rozbudowę i uzyskanie jeszcze większych korzyści. Uniwersalność systemu oraz problem który realizuje nie ogranicza się jedynie do Wydziału Informatyki Politechniki Poznańskiej. System jest w stanie objąć swoim działaniem całą Politechnikę Poznańską, jak również mógłby zostać z powodzeniem wdrożony na innych uczelniach wyższych borykających się z identycznym problemem braku wiedzy na temat swoich absolwentów. Poszerzenie źródeł z których pochodzą dane zdecydowanie polepszyłoby ilość i jakość wiedzy uzyskanej z systemu LinkedInGrads, co z pewnością zwiększy jego wartość biznesową.
	
\section{Propozycja dalszych prac}
\label{Chapter92}

\begin{itemize}
\item Rozszerzenie o inne źródła danych - np. GoldenLine, Profeo.
\item połączenie z eDziekanatem.
\item Rozszerzenie na inne uczelnie, oraz wydziały Politechniki Poznańskiej.
\item Stworzenie automatycznych testów akceptacyjnych dla wszystkich funkcji głównego scenariusza wykorzystując np. Selenium.
\item Podzielenie grupowania pojęć na działy odpowiadające treści którą chcemy grupować.
\item Przegląd zaimportowanych absolwentów i rozszerzenie informacji posiadanej o źródła danych posiadanych wcześniej (Ankiety).
\item Utworzenie nowych raportów i rozszerzenie już istniejących.
\item Graficzna reprezentacja danych i analiza za pomocą wykresów wewnątrz systemu.
\item Łączenie danych kierunków studiów z planami kształcenia.
\item Badanie wpływu ITS na przebieg kariery absolwenta.
\end{itemize}