\chapter{Wymagania pozafunkcjonalne}
\label{Chapter4}

Wymagania pozafunkcjonalne (ang. non-functional requirements) określają jakość sposobu realizacji funkcji przez system. Mimo, że wymagania funkcjonalne uda się spełnić, to nie zawsze realizacja celu jest w stanie usatysfakcjonować odbiorcę systemu. Dobrze sprecyzowane wymagania pozafunkcjonalne pozwalają na uniknięcie niskiej użyteczności systemu oraz mogą mieć wpływ na sam sposób realizacji projektu. Stworzenie i weryfikacja tych wymagań jest ważna zarówno dla zlecającego projekt jak i wykonawcy projektu, gdyż doprecyzowują one użyteczność korzystania z systemu, co pozwala na uniknięcie sporów podczas odbioru projektu, w przypadku niskiej jego jakości. Jakość uzyskanego systemu ma również wpływ na jego późniejsze utrzymanie po stronie klienta, stąd tak ważne rozsądne opisanie wymagań pozafunkcjonalnych, które nie zawsze potrafią być oczywiste dla użytkowników końcowych.

\section{Standard ISO/IEC FDIS 25010}

Model jakości oprogramowania standardu ISO/IEC FDIS 25010 [3] wyróżnia następujące charakterystyki i podcharakterystyki jakości oprogramowania:

\begin{itemize}
\item Funkcjonalne dopasowanie (ang. Functional suitability)
	\begin{itemize}
	\item Funkcjonalna kompletnosc (ang. Functional completeness)
	\item Funkcjonalna odpowiedniość (ang. Functional correctness)
	\item Funkcjonalna poprawność (ang. Functional appropriateness)
	\end{itemize}
\item Wydajność (ang. Performance efficiency)
	\begin{itemize}
	\item Charakterystyka czasowa (ang. Time behaviour)
	\item Zużycie zasobów (ang. Resource utilization)
	\item Oczekiwana wydajność (ang. Capacity)
	\end{itemize}
\item Kompatybilność (ang. Compatibility)
	\begin{itemize}
	\item Współistnienie (ang. Co-existence)
	\item Interoperacyjność (ang. Interoperability)
	\end{itemize}
\item Użyteczność (ang. Usability)
	\begin{itemize}
	\item Rozpoznawalność zastosowania (ang. Appropriateness recognizability)
	\item Łatwość nauczenia się (ang. Learnability)
	\item Łatwość operowania (ang. Operability)
	\item Ochrona użytkownika przed błędami (ang. User error protection)
	\item Estetyka interfejsu użytkownika (ang. User interface aesthetics)
	\item Dostępność personalna (ang. Accessibility)
	\end{itemize}
\item Niezawodność (ang. Reliability)
	\begin{itemize}
	\item Dojrzałość (ang. Maturity)
	\item Tolerancja uszkodzeń (ang. Availability)
	\item Odporność na wady (ang. Fault tolerance)
	\item Odtwarzalność (ang. Recoverability)
	\end{itemize}
\item Bezpieczeństwo (ang. Security)
	\begin{itemize}
	\item Poufność (ang. Confidentiality)
	\item Integralność (ang. Integrity)
	\item Niezaprzeczalność (ang. Non-repudiation)
	\item Identyfikowalność (ang. Accountability)
	\item Autentyczność (ang. Authenticity)
	\end{itemize}
\item Łatwość utrzymania (ang. Maintainability)
	\begin{itemize}
	\item Modułowość (ang. Modularity)
	\item Łatwość ponownego wykorzystania (ang. Reusability)
	\item Łatwość analizy (ang. Analysability)
	\item Łatwość zmiany (ang. Modifiability)
	\item Łatwość testowania (ang. Testability)
	\end{itemize}
\item Przenośność (ang. Portability)
	\begin{itemize}
	\item Łatwość adaptacji (ang. Adaptability)
	\item Łatwość instalacji (ang. Installability)
	\item Łatwość zamiany (ang. Replaceability)
	\end{itemize}
\end{itemize}

Wymienione podcharakterystyki posłużyły jako kategorie wymagań pozafunkcjonalnych w projekcie. Ze względu na specyfikę systemu - aplikacji internetowej niektóre kategorie tego standardu nie zostały wykorzystane podczas prac nad wymaganiami pozafunkcjonalnymi.

\section{Wymagania pozafunkcjonalne i ich weryfikacja}

W kolejnych tablicach przedstawiono wymagania pozafunkcjonalne systemu LinkedInGrads, określonych przy pomocy standardu ISO/IEC FDIS 25010. Priorytet wymagań określono za pomocą notacji:

\begin{itemize}
\item Would - System może spełniać dane wymaganie,
\item Should - System powinien spełniać dane wymaganie,
\item Must - System musi spełniać dane wymaganie.
Dodatkowo określono złożoność każdego z wymagań wykorzystując następującą notację:
\item Low - Niski stopień złożoności wymagania pozafunkcjonalnego,
\item Medium - Średni stopień złożoności wymagania pozafunkcjonalnego,
\item High - Wysoki stopień złożoności wymagania pozafunkcjonalnego.
Metryki te pozwalały podczas prac nad projektem na realizację najważniejszych wymagań, uwzględniając ich złożoność czasową skonfrontowaną z dostępnym czasem na realizację projektu. Ostatnia kolumna tabeli określa czy w projekcie LinkedInGrads udało się spełnić dane wymaganie pozafunkcjonalne.
\end{itemize}

\subsection{Funkcjonalne dopasowanie: Funkcjonalna kompletność}

\begin{table}[H]
\centering
\begin{tabular}{ | p{8cm} | c | c | c | }
\hline
\textbf{Wymaganie} & \textbf{Priorytet} & \textbf{Złożoność} & \textbf{Realizacja} \\ \hline
Współpraca z przeglądarką Firefox (dla dwóch najnowszych wersji). & Must & Low & Zrealizowano \\ \hline
\end{tabular}
\caption{Funkcjonalne dopasowanie: Funkcjonalna kompletność}\label{tab:reqs}
\end{table}

System był tworzony wykorzystując do testów najnowszą wersję przeglądarki Mozilla Firefox, oraz dodatkowo Google Chrome. Zapewniło to kompatybilność systemu z tymi przeglądarkami, a funkcje użyte wewnątrz systemu nie wykraczają poza zakres możliwości poprzednich kilku wersji tych przeglądarek.

\subsection{Wydajność: Charakterystyka Czasowa}

\begin{table}[H]
\centering
\begin{tabular}{ | p{8cm} | c | c | c | }
\hline
\textbf{Wymaganie} & \textbf{Priorytet} & \textbf{Złożoność} & \textbf{Realizacja} \\ \hline
Dla 20000 absolwentów aktualizacja danych
z LinkedIn nie powinna trwać dłużej niż
tydzień. & Must & Medium & Zrealizowano \\ \hline
System powinien estymować czas
aktualizacji danych. & Would & High & Pominęto \\ \hline
Generowanie raportu nie może trwać dłużej
niż 10 sekund - w przeciwnym wypadku
informacja o czasie. & Should & Medium & Zrealizowno \\ \hline
\end{tabular}
\caption{Wydajność: Charakterystyka Czasowa}\label{tab:reqs}
\end{table}

Wymagania tej kategorii udało się spełnić, a dzięki aktualizacji danych w tle, estymacja czasu zakończenia aktualizacji okazała się zbędna w finalnej wersji produktu, co również spowodowało niski czas generowania raportu.

\subsection{Kompatybilność: Współistnienie}

\begin{table}[H]
\centering
\begin{tabular}{ | p{8cm} | c | c | c | }
\hline
\textbf{Wymaganie} & \textbf{Priorytet} & \textbf{Złożoność} & \textbf{Realizacja} \\ \hline
Na jednym serwerze może być
uruchomionych wiele instancji aplikacji. & Would & Medium & Zrealizowano \\ \hline
\end{tabular}
\caption{Kompatybilność: Współistnienie}\label{tab:reqs}
\end{table}

Wymaganie to zostało zrealizowane na poziomie architektury poprzez wykorzystanie aplikacji internetowej uruchamianej w środowisku serwera WWW.

\subsection{Kompatybilność: Interoperacyjność}

\begin{table}[H]
\centering
\begin{tabular}{ | p{8cm} | c | c | c | }
\hline
\textbf{Wymaganie} & \textbf{Priorytet} & \textbf{Złożoność} & \textbf{Realizacja} \\ \hline
System ma udostępniać dane absolwenta w
trybie odczytu w formacie JSON przez webservice.
 & Should & High & Pominięto \\ \hline
System ma generować raporty w formacie
CSV.
 & Should & Low & Pominęto \\ \hline
System ma generować raporty w formacie
XLS.
 & Must & Low & Zrealizowano \\ \hline
\end{tabular}
\caption{Kompatybilność: Interoperacyjność}\label{tab:reqs}
\end{table}

Ze względu na duży narzut czasowy wymagań pozafunkcjonalnych z tej kategorii udało się zrealizować tylko generowanie raportu w formacie XLS. Pominięte wymagania funkcjonalne nie były kluczowe dla systemu, co spowodowało, że ich realizacja nie została przydzielona do żadnego ze sprintów.

\subsection{Użyteczność: Łatwość nauczenia się}

\begin{table}[H]
\centering
\begin{tabular}{ | p{8cm} | c | c | c | }
\hline
\textbf{Wymaganie} & \textbf{Priorytet} & \textbf{Złożoność} & \textbf{Realizacja} \\ \hline
Interfejs użytkownika powinien być skonsultowany ze wszystkimi potencjalnymi użytkownikami.
 & Must & Low & Zrealizowano \\ \hline
Instrukcja użytkownika powinna być zorientowana na cele poszczególnych aktorów.
 & Must & Low & Zrealizowano \\ \hline
\end{tabular}
\caption{Użyteczność: Łatwość nauczenia się}\label{tab:reqs}
\end{table}

Zespół był wysoce zorientowany na potrzeby użytkownika końcowego, co już w samej fazie projektowania interfejsów użytkownika zapewniło wysoką jakość systemu. Po skonsultowaniu z użytkownikami, wdrożono udoskonalenia co pozwoliło na zrealizowanie wymagania pozafunkcjonalnego.

\subsection{Użyteczność: Ochrona użytkownika przed błędami}

\begin{table}[H]
\centering
\begin{tabular}{ | p{8cm} | c | c | p{2cm} | }
\hline
\textbf{Wymaganie} & \textbf{Priorytet} & \textbf{Złożoność} & \textbf{Realizacja} \\ \hline
System musi być przygotowany na rozszerzenie list atrybutów.
 & Must & Low & Zrealizowano \\ \hline
System powinien logować błedy (wyjątki) do logu systemowego.
 & Must & Low & Zrealizowano \\ \hline
 System powinien informować o błędnej konfiguracji natychmiast po jej wprowadzeniu. & Would & Medium & Pominięto \\ \hline
 System ma potwierdzać powiązanie/odwiązanie użytkownika. & Would & Low & Pominięto \\ \hline
 System powinien zapewnić kontrolę nieprzewidzianych wyjątków. & Must & Low & Zrealizowano częściowo. \\ \hline
 Po wykonaniu importu danych system informuje ilu absolwentów zaimportowano, a ile jest konfliktów. & Must & Low & Pominięto \\ \hline
\end{tabular}
\caption{Użyteczność: Ochrona użytkownika przed błędami}\label{tab:reqs}
\end{table}

System został przygotowany mając od początku na uwadze możliwość rozszerzenia go o  dodatkowe dane jak i dodatkowe źródła danych, co zapewniło realizację pierwszego z wymagań. Wszystkie błędy które pojawią się w aplikacji automatycznie zapisywane są do logów, co częściowo realizuje również wymaganie dotyczące kontroli nieprzewidzianych wyjątków, które zostają w momencie wystąpienia ukryte przed użytkownikiem końcowym. Podczas prac nad systemem zrezygnowano z możliwości konfiguracji parametrów jego pracy, ze względu na brak zasadności jakichkolwiek zmian. Wymaganie to powstało mając na uwadze możliwe ograniczenia narzucone na system pochodzące z zewnętrznych źródeł danych. Również w trakcie prac zrezygnowano z automatycznego wyszukiwania użytkowników wewnątrz sieci społecznościowej LinkedIn i wiązania znalezionych kont z danymi osób dostarczonymi z dziekanatu. Wyeliminowało to potrzebę potwierdzania powiązania/odwiązania użytkownika, gdyż dane te wprowadzane są manualnie przez użytkownika. Ostatnie z wymagań nie zostało zrealizowane ze względu na dołączenie go do wymagań pozafunkcjonalnych w późnej fazie projektu i nie uwzględnienie jego realizacji w żadnym ze sprintów.

\subsection{Użyteczność: Estetyka interfejsu użytkownika}

\begin{table}[H]
\centering
\begin{tabular}{ | p{8cm} | c | c | c | }
\hline
\textbf{Wymaganie} & \textbf{Priorytet} & \textbf{Złożoność} & \textbf{Realizacja} \\ \hline
Interfejs powinien zawierać logo Politechniki Poznańskiej oraz opis przeznaczenia systemu.
 & Must & Low & Zrealizowano \\ \hline
\end{tabular}
\caption{Użyteczność: Estetyka interfejsu użytkownika}\label{tab:reqs}
\end{table}

\subsection{Użyteczność: Dostępność personalna}

\begin{table}[H]
\centering
\begin{tabular}{ | p{8cm} | c | c | c | }
\hline
\textbf{Wymaganie} & \textbf{Priorytet} & \textbf{Złożoność} & \textbf{Realizacja} \\ \hline
System dostępny tylko w jednej wersji językowej - Polskiej
 & Must & Low & Zrealizowano \\ \hline
\end{tabular}
\caption{Użyteczność: Dostępność personalna}\label{tab:reqs}
\end{table}

\subsection{Niezawodność: Tolerancja uszkodzeń}

\begin{table}[H]
\centering
\begin{tabular}{ | p{8cm} | c | c | c | }
\hline
\textbf{Wymaganie} & \textbf{Priorytet} & \textbf{Złożoność} & \textbf{Realizacja} \\ \hline
Przerwy serwisowe możliwe są w godzinach 18-6 lub po wcześniejszym ustaleniu.
 & Should & Low & Pominięto \\ \hline
 Niekontrolowana awaria może trwać maksymalnie 2 dni. & Should & Low & Pominięto \\ \hline
\end{tabular}
\caption{Niezawodność: Tolerancja uszkodzeń}\label{tab:reqs}
\end{table}

Wymagania należące do tej kategorii nie zostały zrealizowane. Wynika to z faktu, że dotyczą one utrzymania systemu po jego wdrożeniu, więc weryfikacja tych wymagań była niemożliwa do przeprowadzenia.	

\subsection{Niezawodność: Odporność na wady}

\begin{table}[H]
\centering
\begin{tabular}{ | p{8cm} | c | c | c | }
\hline
\textbf{Wymaganie} & \textbf{Priorytet} & \textbf{Złożoność} & \textbf{Realizacja} \\ \hline
System należy poddać analizie z wykorzystaniem metody ATAM.
 & Must & Low & Zrealizowano \\ \hline
Wszystkie funkcje głównego scenariusza powinny być możliwe do przetestowania w sposób automatyczny.
 & Should & Low & Zrealizowano \\ \hline
W przypadku problemów z połączeniem, proces aktualizacji danych powinien być wznawiany (od stanu w którym nastąpiło przerwanie). & Would & Low & Zrealizowano \\ \hline
System powinien posiadać testy jednostkowe oraz testy akceptacyjne. & Must & High & Zrealizowano \\ \hline
W przypadku problemów z połączeniem proces aktualizacji danych powinien być ponawiany. & Must & Low & Zrealizowano \\ \hline
\end{tabular}
\caption{Niezawodność: Odporność na wady}\label{tab:reqs}
\end{table}

\subsection{Niezawodność: Odtwarzalność}

\begin{table}[H]
\centering
\begin{tabular}{ | p{8cm} | c | c | c | }
\hline
\textbf{Wymaganie} & \textbf{Priorytet} & \textbf{Złożoność} & \textbf{Realizacja} \\ \hline
System powinien tworzyć kopie zapasowe danych codziennie.
 & Must & Low & Zrealizowano \\ \hline
\end{tabular}
\caption{Niezawodność: Odtwarzalność}\label{tab:reqs}
\end{table}

\subsection{Bezpieczeństwo: Poufność}

\begin{table}[H]
\centering
\begin{tabular}{ | p{8cm} | c | c | c | }
\hline
\textbf{Wymaganie} & \textbf{Priorytet} & \textbf{Złożoność} & \textbf{Realizacja} \\ \hline
System zapewnia szyfrowane połączenie z użytkownikiem (nie wymaga zaufanego certyfikatu). 
 & Should & Low & Zrealizowano \\ \hline
System powinien korzystać z systemu eLogin do uwierzytelniania użytkowników. & Must & Medium & Zrealizowano \\ \hline
\end{tabular}
\caption{Bezpieczeństwo: Poufność}\label{tab:reqs}
\end{table}

\subsection{Bezpieczeństwo: Integralność}

\begin{table}[H]
\centering
\begin{tabular}{ | p{8cm} | c | c | c | }
\hline
\textbf{Wymaganie} & \textbf{Priorytet} & \textbf{Złożoność} & \textbf{Realizacja} \\ \hline
Dane przechowywane w systemie powinny być chronione przed nieuprawnionym odczytem, modyfikacją, usunięciem.
 & Must & Low & Zrealizowano \\ \hline
System powinien być odporny na SQL injection. & Must & Low & Zrealizowano \\ \hline
\end{tabular}
\caption{Bezpieczeństwo: Integralność}\label{tab:reqs}
\end{table}

Wymagania zawarte w tej kategorii zostały zrealizowane na poziomie architektury, poprzez wykorzystanie przygotowanych zapytań (ang. prepared statements), oraz wymaganie uwierzytelnienia użytkownika.

\subsection{Bezpieczeństwo: Niezaprzeczalność}

\begin{table}[H]
\centering
\begin{tabular}{ | p{8cm} | c | c | c | }
\hline
\textbf{Wymaganie} & \textbf{Priorytet} & \textbf{Złożoność} & \textbf{Realizacja} \\ \hline
Import danych powinien mieć stempel czasowy – w raporcie powinna się znaleźć data aktualizacji danych na podstawie których będzie generowany raport.
 & Must & Low & Zrealizowano \\ \hline
\end{tabular}
\caption{Bezpieczeństwo: Niezaprzeczalność}\label{tab:reqs}
\end{table}

\subsection{Łatwość utrzymania: Łatwość analizy}

\begin{table}[H]
\centering
\begin{tabular}{ | p{8cm} | c | c | c | }
\hline
\textbf{Wymaganie} & \textbf{Priorytet} & \textbf{Złożoność} & \textbf{Realizacja} \\ \hline
Należy zapewnić dokumentację zgodną z wymaganiami DRO.
 & Must & Medium & Zrealizowano \\ \hline
Kod źródłowy systemu musi przestrzegać konwencji wymaganej przez DRO. & Must & Medium & Zrealizowano \\ \hline
\end{tabular}
\caption{Łatwość utrzymania: Łatwość analizy}\label{tab:reqs}
\end{table}

\subsection{Łatwość utrzymania: Łatwość zmiany}

\begin{table}[H]
\centering
\begin{tabular}{ | p{8cm} | c | c | c | }
\hline
\textbf{Wymaganie} & \textbf{Priorytet} & \textbf{Złożoność} & \textbf{Realizacja} \\ \hline
Architektura powinna zapewnić rozdzielenie warstwy backendowej od frontendowej.
 & Must & Low & Zrealizowano \\ \hline
System musi być przygotowany na zmianę generowania raportów i udostępniania danych. & Must & Low & Zrealizowano \\ \hline
Model danych systemu powinien być przygotowany na zmiany: integrację z nowymi systemami, adaptację na potrzeby innych uczelni. & Must & Low & Zrealizowano \\ \hline
\end{tabular}
\caption{Łatwość utrzymania: Łatwość zmiany}\label{tab:reqs}
\end{table}

Rozdzielenie warstwy frontendowej od backendowej zostało zrealizowane z pomocą frameworka Ruby on Rails implementującego wzorzec projektowy MVC (Model-view-controller). System zapewnia również możliwość tworzenia nowych raportów, oraz integrację z innymi systemami ze względu na modularną architekturę systemu.

\subsection{Łatwość utrzymania: Łatwość testowania}

\begin{table}[H]
\centering
\begin{tabular}{ | p{8cm} | c | c | c | }
\hline
\textbf{Wymaganie} & \textbf{Priorytet} & \textbf{Złożoność} & \textbf{Realizacja} \\ \hline
Przewidzenie dwóch trybów - produkcyjny i testowy.
 & Should & Low & Zrealizowano \\ \hline
\end{tabular}
\caption{Łatwość utrzymania: Łatwość testowania}\label{tab:reqs}
\end{table}

\subsection{Przenośność: Łatwość instalacji}

\begin{table}[H]
\centering
\begin{tabular}{ | p{8cm} | c | c | c | }
\hline
\textbf{Wymaganie} & \textbf{Priorytet} & \textbf{Złożoność} & \textbf{Realizacja} \\ \hline
Należy dostarczyć szczegółową instrukcję instalacji oraz konfiguracji środowiska wykonawczego.
 & Must & Medium & Zrealizowano \\ \hline
\end{tabular}
\caption{Przenośność: Łatwość instalacji}\label{tab:reqs}
\end{table}