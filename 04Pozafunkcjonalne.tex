\chapter{Wymagania pozafunkcjonalne}
\label{Chapter4}

\section{Wstęp}
\label{Chapter41}

W niniejszym rozdziale zostaną zaprezentowane i krótko opisane charakterystyki oraz wymagania pozafunkcjonalne obowiązujące dla systemu. Ponadto zostanie podjęta próba weryfikacji, które wymagania udało się spełnić i jakie są perspektywy rozwoju

\section{Charakterystyki oprogramowania}

Poniżej są jeszcze stare charakterystyki oprogramowania (wtedy myśleliśmy, że to mądrze brzmi), kategorii wymagań pozafunkcjonalnych. Obecnie to się trochę zmieniło, zatem ta lista będzie znacznie bardziej rozbudowana.

\begin{itemize}
\item Dokładność (ang. \definicja{accuracy})
\item Bezpieczeństwo (ang. \definicja{security})
\item Odporność na błędy (ang. \definicja{fault tolerance})
\item Odtwarzalność (ang. \definicja{recoverability})
\item Charakterystyka czasowa (ang. \definicja{time behaviour})
\item Łatwość analizowania (ang. \definicja{analysability})
\item Łatwość zmian (ang. \definicja{changeability})
\item Adaptowalność (ang. \definicja{adaptability})
\item Instalowalność (ang. \definicja{installability})
\item Współistnienie (ang. \definicja{co-existence})
\item Zamienność (ang. \definicja{replaceability})
\end{itemize}

Tutaj piszemy, które podcharakterystyki były dla nas priorytetowe lub szczególnie ważne, a które mniej. Oczywiście, z uzasadnieniem. Piszemy również, jak zamierzaliśmy (lub to robiliśmy) dbać o to, aby wszystko było spełnione.

\section{Wymagania pozafunkcjonalne i ich weryfikacja}

W tablicy \ref{tab:reqs} przedstawiono wymagania pozafunkcjonalne związane z systemem. W kolumnach \textbf{Priorytet} oraz \textbf{Trudność} określono poziomy przy pomocy następującej notacji:

\begin{itemize}
\item H -- wysoki priorytet lub poziom trudności
\item M -- średni priorytet lub poziom trudności
\item L -- niski priorytet lub poziom trudności
\item N -- wymaganie oczywiste lub bardzo proste do spełnienia
\end{itemize}
\begin{table}[h]
\centering
\begin{tabular}{ | c | p{7cm} | c | c | }
\hline
\textbf{Podcharakterystyka} & \textbf{Wymaganie} & \textbf{Priorytet} & \textbf{Trudność} \\ \hline
Nazwa podcharakterystyki & Wymaganie 1 (przykład procentów: 90\%) & H & H \\ \hline
Nazwa podcharakterystyki & Wymaganie 2 & M & H \\ \hline
Nazwa podcharakterystyki & Wymaganie 3 & H & L \\ \hline
Nazwa podcharakterystyki & Wymaganie 4 & N & L \\ \hline
... & ... & ... & ... \\ \hline
\end{tabular}
\caption{Wymagania pozafunkcjonalne}\label{tab:reqs}
\end{table}
A tutaj piszemy o wszelkich problemach, wszelkich naszych wnioskach związanych ze spełnianiem wymagań pozafunkcjonalnych -- co się udało, co nie (i dlaczego). Tak, jakbyśmy opisywali nasze doświadczenia i problemy, z jakimi przyszło nam się zmagać i być może rozwiązać. Staramy się odnieść do najważniejszych wymagań (chyba że jest ich mało, wtedy do wszystkich).