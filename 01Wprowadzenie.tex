\chapter{Wprowadzenie}
\label{Chapter1}

\section{Opis problemu i koncepcja jego rozwiązania}
\label{Chapter11}

Zgodnie z rozporządzeniem Ministerstwa Nauki i Szkolnictwa Wyższego [1,2] każda uczelnia wyższa w Polsce jest zobowiązana do monitorowania karier swoich absolwentów, w celu pozyskania informacji o ich aktualnej sytuacji zawodowej. Oprócz spełnienia wymogów formalnych, systematyczne gromadzenie i analizowanie danych o zatrudnieniu absolwentów umożliwia uczelniom weryfikację jakości i efektywności kształcenia na poszczególnych wydziałach i kierunkach. Zebrane informacje stanowią cenną wskazówkę w ciągłym procesie doskonalenia oferty dydaktycznej uczelni, pomagając w dostosowywaniu kierunków studiów i programów kształcenia do potrzeb rynku pracy. Prowadzenie rzetelnych badań na temat losów zawodowych absolwentów i prezentowanie statystyk zatrudnienia jest również istotne z punktu widzenia wizerunku uczelni.

Ministerstwo Nauki i Szkolnictwa Wyższego nie narzuca uczelniom sposobu realizacji procesu monitorowania karier absolwentów. Większość uczelni wyższych, w tym Politechnika Poznańska, wywiązuje się z tego obowiązku za pomocą badania ankietowego. Opracowane ankiety są rozsyłane do absolwentów w formie elektronicznych lub drukowanych formularzy bądź przeprowadzane za pośrednictwem rozmów telefonicznych. Rozwiązania te są nie tylko kosztowne i czasochłonne, lecz charakteryzują się także niewielką efektywnością. Z szacunków Centrum Praktyk i Karier Politechniki Poznańskiej wynika,, że z możliwości dobrowolnego wypełnienia ankiety absolwenckiej korzysta poniżej 10% absolwentów.

W związku z wymienionymi wadami dotychczasowych metod zaproponowano stworzenie systemu informatycznego, w postaci aplikacji internetowej, który monitorowałby kariery zawodowe absolwentów wykorzystując dane udostępniane przez nich w serwisie LinkedIn. Serwis LinkedIn stanowi jedną z największych sieci zawodowych, łącząc ponad 250 mln. użytkowników w 200 krajach i terytoriach na całym świecie. Stworzony system powinien w założeniach zautomatyzować i uskutecznić proces monitorowania, pobierać dane o zatrudnieniu absolwentów z serwisu LinkedIn oraz prezentować je w formie raportów o z góry określonej strukturze. Należy przewidzieć również możliwość integracji z innymi serwisami mogącymi posłużyć jako źródło danych o sytuacji zawodowej absolwentów.

System został zrealizowany na Wydziale Informatyki Politechniki Poznańskiej w ramach zajęć Studia Rozwoju Oprogramowania. Wykonanie systemu zostało zlecone przez rzeczywistego klienta, w postaci przedstawicieli władz wydziału i uczelni. Prace były prowadzone według przyjętej metodyki i harmonogramu.

\section{Omówienie pracy}
\label{Chapter12}

Niniejsza praca opisuje otwarty system monitorowania karier zawodowych absolwentów LinkedInGrads (ang. LinkedInGrads: graduate career tracking system), zwany dalej Systemem, realizujący koncepcję przedstawioną w punkcie 1.1.  Praca stanowi dokumentację techniczną systemu, a także wyjaśnia idee stojące za poszczególnymi decyzjami projektowymi. Powinna być przydatna zarówno dla użytkowników końcowych systemu, jak i dla osób, które zamierzają go wdrożyć, utrzymywać bądź rozwijać. Jako praca dyplomowa inżynierska jest również skierowana do członków komisji egzaminacyjnej.

W rozdziale 2. przedstawiono aktorów, obiekty biznesowe oraz przypadki użycia występujące w systemie. W rozdziale 3. opisano wymagania funkcjonalne, a w rozdziale 4. wymagania pozafunkcjonalne, wraz z informacją, które z nich zostały zrealizowane. W rozdziale 5. omówiono ogólną architekturę systemu. Rozdział 6. zawiera szczegóły implementacji systemu oraz opis wykorzystanych koncepcji i technologii. W rozdziale 7. przedstawiono metody i narzędzia wspomagające zapewnienie jakości systemu. W rozdziale 8. opisano zebrane wnioski i doświadczenia. Rozdział 9. zawiera podsumowanie całości projektu oraz propozycje dalszego rozwoju systemu. W skład dokumentu wchodzi również bibliografia pracy oraz dodatki, obejmujące informacje uzupełniające, prezentację wyglądu aplikacji, instrukcję instalacji oraz scenariusze manualnych testów akceptacyjnych.