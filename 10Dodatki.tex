\chapter{Informacje uzupełniające}
\label{Chapter10}

\section{Wkład poszczególnych osób do przedsięwzięcia}
\label{Chapter101}

Skład zespołu pracującego nad projektem został przedstawiony w tablicy \ref{tab:roster}.

\begin{table}[h]
\centering
\begin{tabular}{ | c | c | }
\hline
\textbf{Stanowisko} & \textbf{Osoba} \\ \hline
Założyciel projektu, klient & Tytuł Imię Nazwisko \\ \hline
Główny użytkownik & Tytuł Imię Nazwisko \\ \hline
Główny dostawca & Tytuł Imię Nazwisko \\ \hline
Dostawca od strony DRO & Tytuł Imię Nazwisko \\ \hline
Starszy konsultant & Tytuł Imię Nazwisko \\ \hline
Konsultant & Tytuł Imię Nazwisko \\ \hline
Kierownik projektu & inż.~Imię Nazwisko \\ \hline
Analityk/Architekt & inż.~Imię Nazwisko \\ \hline
Programiści & Imię Nazwisko \\ 
 & Imię Nazwisko \\ 
 & Imię Nazwisko \\ 
 & Imię Nazwisko \\
\hline
\end{tabular}
\caption{Osoby związane z przedsięwzięciem}\label{tab:roster}
\end{table}

\noindent
Teraz bardzo ważna rzecz -- w tym miejscu piszecie, co kto przygotowywał w tekście pracy inżynierskiej. Prawdopodobnie tutaj będziecie musieli wymienić, które rozdziały został dla Was przygotowane przez kierownika projektu, analityka, architekta lub inną osobę. Oczywiście, piszecie też, za które części dokumentu Wy jesteście odpowiedzialni. To jest ważna, aby ta część była tutaj precyzyjnie przygotowana -- takie są wymogi uczelni oraz też uwzględnienia pracy innych osób.

\noindent

Odpowiedzialność za część implementacyjną systemu została przedstawiona poniżej:

\begin{description}
\item Imię i nazwisko pierwszego programisty

\begin{itemize}
\item Odpowiedzialność 1
\item Odpowiedzialność 2
\item Odpowiedzialność 3
\item ...
\end{itemize}
\noindent

\item Imię i nazwisko drugiego programisty

\begin{itemize}
\item Odpowiedzialność 1
\item Odpowiedzialność 2
\item Odpowiedzialność 3
\item ...
\end{itemize}
\noindent

\item Imię i nazwisko trzeciego programisty

\begin{itemize}
\item Odpowiedzialność 1
\item Odpowiedzialność 2
\item Odpowiedzialność 3
\item ...
\end{itemize}
\noindent

\item Imię i nazwisko czwartego programisty

\begin{itemize}
\item Odpowiedzialność 1
\item Odpowiedzialność 2
\item Odpowiedzialność 3
\item ...
\end{itemize}
\noindent

\end{description}

\noindent
Ewentualne podziękowania dla innych osób, które Wam pomagały, mowy dziękczynne, itd.

\section{Wykaz użytych narzędzi}
\label{Chapter104}

Wprawdzie jest odpowiedni podrozdział w rozdziale \ref{Chapter5}, ale tutaj można wymienić nawet małe narzędzia i biblioteki, które wykorzystywaliście (np.~narzędzie do robienia makiet interfejsu) i które warto wymienić, także dla przyszłych roczników (można też dać linki).

\section{Zawartość płyty CD}

Do dokumentu załączono płytę CD o następującej zawartości:

\begin{itemize}
\item Zawartość 1
\item Zawartość 2
\item Zawartość 3
\item ...
\end{itemize}