\chapter{Scenariusze manualnych testów akceptacyjnych}

\begin{tabular}{ |p{0.8cm}|p{6cm}|p{6cm}| }
\hline
\multicolumn{3}{ |p{12.8cm}| }{MAT01: Generowanie raportu o absolwentach} \\
\hline

\multicolumn{3}{ |p{12.8cm}| }{Warunki początkowe: 

\begin{itemize}
\item Użytkownik jest zalogowany w systemie eLogin.
\item Użytkownik ma uprawnienia do korzystania z Systemu.
\end{itemize}

} \\
\hline

 Krok & Akcja & Oczekiwana odpowiedź \\ \hline
 1. & Użytkownik wybiera z głównego menu opcję Raporty. & System wyświetla listę dostępnych raportów. \\ \hline
 2. & Użytkownik wybiera raport z listy dostępnych raportów. & System wyświetla formularz konfiguracji raportu. \\ \hline
 3. & Użytkownik wypełnia wyświetlony formularz konfiguracji raportu. & - \\ \hline
 4. & Użytkownik naciska przycisk Zapisz. & System generuje wybrany raport i udostępnia go Użytkownikowi. \\ \hline
\multicolumn{3}{ |p{12.8cm}| }{Uwagi: Sposób udostępniania Użytkownikowi wygenerowanego przez System raportu jest zależny od ustawień przeglądarki Użytkownika. Standardowa realizacja polega na wyświetleniu okna modalnego umożliwiającego potwierdzenie chęci zapisania pliku raportu na dysku twardym Użytkownika.} \\
\hline 
\end{tabular}

\begin{tabular}{ |p{0.8cm}|p{6cm}|p{6cm}| }
\hline
\multicolumn{3}{ |p{12.8cm}| }{MAT02: Tworzenie grupy} \\
\hline

\multicolumn{3}{ |p{12.8cm}| }{Warunki początkowe: 

\begin{itemize}
\item Użytkownik jest zalogowany w systemie eLogin.
\item Użytkownik ma uprawnienia do korzystania z Systemu.
\end{itemize}

} \\
\hline

 Krok & Akcja & Oczekiwana odpowiedź \\ \hline
 1. & Użytkownik wybiera z głównego menu opcję Grupowanie. & System wyświetla widok grupowania pojęć. \\ \hline
 2. & Użytkownik wpisuje nazwę nowej grupy w pole Nowa Grupa. & - \\ \hline
 3. & Użytkownik wypełnia wyświetlony formularz konfiguracji raportu. & - \\ \hline
 4. & Użytkownik naciska przycisk oznaczony ikoną plusa lub klawisz Enter. & System dodaje do listy grup nową grupę o wpisanej nazwie lub, jeśli grupa o wpisanej nazwie już istnieje, wyświetla okno modalne z informacją o niepowodzeniu. \\ \hline
\multicolumn{3}{ |p{12.8cm}| }{empty} \\
\hline 
\end{tabular}

\begin{tabular}{ |p{0.8cm}|p{6cm}|p{6cm}| }
\hline
\multicolumn{3}{ |p{12.8cm}| }{MAT03: Przypisanie pojęcia do grupy} \\
\hline

\multicolumn{3}{ |p{12.8cm}| }{Warunki początkowe: 

\begin{itemize}
\item Użytkownik jest zalogowany w systemie eLogin.
\item Użytkownik ma uprawnienia do korzystania z Systemu.
\item W Systemie istnieje co najmniej jedna grupa.
\item W Systemie istnieje co najmniej jedno pojęcie nieprzypisane do żadnej grupy.
\end{itemize}

} \\
\hline

 Krok & Akcja & Oczekiwana odpowiedź \\ \hline
 1. & Użytkownik wybiera z głównego menu opcję Grupowanie. & System wyświetla widok grupowania pojęć. \\ \hline
 2. & Użytkownik wybiera z listy grup wybraną grupę. & System wyświetla listę pojęć przypisanych do wybranej grupy oraz listę pojęć nieprzypisanych do żadnej grupy.
 \\ \hline
 3. & Użytkownik naciska przycisk oznaczony ikoną plusa przy wybranym pojęciu z listy pojęć nieprzypisanych do żadnej grupy. & System przenosi wybrane pojęcie z listy pojęć nieprzypisanych do żadnej grupy do listy pojęć przypisanych do wybranej grupy.  \\ \hline
 \\ \hline
\multicolumn{3}{ |p{12.8cm}| }{empty} \\
\hline 
\end{tabular}

\begin{tabular}{ |p{0.8cm}|p{6cm}|p{6cm}| }
\hline
\multicolumn{3}{ |p{12.8cm}| }{MAT04: Usuwanie grupy} \\
\hline

\multicolumn{3}{ |p{12.8cm}| }{Warunki początkowe: 

\begin{itemize}
\item Użytkownik jest zalogowany w systemie eLogin.
\item Użytkownik ma uprawnienia do korzystania z Systemu.
\item W Systemie istnieje co najmniej jedna grupa.
\end{itemize}

} \\
\hline

 Krok & Akcja & Oczekiwana odpowiedź \\ \hline
 1. & Użytkownik wybiera z głównego menu opcję Grupowanie. & System wyświetla widok grupowania pojęć. \\ \hline
 2. & Użytkownik wybiera z listy grup wybraną grupę. & System wyświetla listę pojęć przypisanych do wybranej grupy oraz listę pojęć nieprzypisanych do żadnej grupy.
 \\ \hline
 3. & Użytkownik naciska przycisk oznaczony ikoną kosza. & System wyświetla okno modalne z potwierdzeniem wykonania operacji.  \\ \hline
 4. & Użytkownik potwierdza wykonanie operacji naciskając przycisk OK. & System usuwa wybraną grupę i przenosi przypisane do niej pojęcia do listy pojęć nieprzypisanych do żadnej grupy. \\ \hline
\multicolumn{3}{ |p{12.8cm}| }{empty} \\
\hline 
\end{tabular}

\begin{tabular}{ |p{0.8cm}|p{6cm}|p{6cm}| }
\hline
\multicolumn{3}{ |p{12.8cm}| }{MAT05: Usuwanie pojęcia z grupy} \\
\hline

\multicolumn{3}{ |p{12.8cm}| }{Warunki początkowe: 

\begin{itemize}
\item Użytkownik jest zalogowany w systemie eLogin.
\item Użytkownik ma uprawnienia do korzystania z Systemu.
\item W Systemie istnieje co najmniej jedna grupa.
\item W Systemie istnieje co najmniej jedno pojęcie przypisane do grupy.
\end{itemize}

} \\
\hline

 Krok & Akcja & Oczekiwana odpowiedź \\ \hline
 1. & Użytkownik wybiera z głównego menu opcję Grupowanie. & System wyświetla widok grupowania pojęć. \\ \hline
 2. & Użytkownik wybiera z listy grup wybraną grupę. & System wyświetla listę pojęć przypisanych do wybranej grupy oraz listę pojęć nieprzypisanych do żadnej grupy.
 \\ \hline
 3. & Użytkownik naciska przycisk oznaczony ikoną minusa przy wybranym pojęciu z listy pojęć przypisanych do wybranej grupy. & System przenosi wybrane pojęcie z listy pojęć przypisanych do wybranej grupy do listy pojęć nieprzypisanych do żadnej grupy. \\ \hline
\multicolumn{3}{ |p{12.8cm}| }{empty} \\
\hline 
\end{tabular}

\begin{tabular}{ |p{0.8cm}|p{6cm}|p{6cm}| }
\hline
\multicolumn{3}{ |p{12.8cm}| }{MAT06: Zmiana nazwy grupy} \\
\hline

\multicolumn{3}{ |p{12.8cm}| }{Warunki początkowe: 

\begin{itemize}
\item Użytkownik jest zalogowany w systemie eLogin.
\item Użytkownik ma uprawnienia do korzystania z Systemu.
\item W Systemie istnieje co najmniej jedna grupa.
\end{itemize}

} \\
\hline

 Krok & Akcja & Oczekiwana odpowiedź \\ \hline
 1. & Użytkownik wybiera z głównego menu opcję Grupowanie. & System wyświetla widok grupowania pojęć. \\ \hline
 2. & Użytkownik wybiera z listy grup wybraną grupę. & System wyświetla listę pojęć przypisanych do wybranej grupy oraz listę pojęć nieprzypisanych do żadnej grupy.
 \\ \hline
 3. & Użytkownik naciska przycisk oznaczony ikoną ołówka. & System wyświetla nazwę wybranej grupy w polu Nowa Nazwa Grupy. \\ \hline
 4. & Użytkownik wpisuje nową nazwę wybranej grupy w pole Nowa Nazwa Grupy. & - \\ \hline
 5. & Użytkownik naciska przycisk oznaczony ikoną ołówka lub klawisz Enter. & System zmienia nazwę wybranej grupy na wpisaną przez Użytkownika lub, jeśli grupa o wpisanej nazwie już istnieje, wyświetla okno modalne z informacją o niepowodzeniu. \\ \hline
\multicolumn{3}{ |p{12.8cm}| }{empty} \\
\hline 
\end{tabular}
