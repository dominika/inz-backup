\chapter{Zapewnianie jakości i konserwacja systemu}
\label{Chapter7}

Podstawą zapewniania jakości każdego systemu informatycznego jest wprowadzenie i pielęgnacja mechanizmów umożliwiających jego systematyczną weryfikację i walidację. Obecność takich mechanizmów w projekcie pozwala na możliwie wczesne wykrywanie i korygowanie błędów związanych z niezgodności systemu ze specyfikacją wymagań bądź z potrzebami i oczekiwaniami klienta. W systemie LinkedIGrads głównymi mechanizmami zapewniania jakości było stworzenie zestawu testów automatycznych i manualnych, wykorzystanie systemu do ciągłej integracji, przestrzeganie przyjętych standardów kodowania oraz realizacja projektu zgodnie z modelem przyrostowym.

\section{Testy i weryfikacja jakości oprogramowania}
\label{Chapter71}

\subsection{Środowisko testowe}

Do napisania testów automatycznych systemu wykorzystano następujące narzędzia:

\begin{itemize}
\item RSpec Rails - framework do tworzenia i wykonywania testów aplikacji Ruby on Rails, w myśli idei programowania sterowanego zachowaniami (ang. Behaviour Driven Development),
\item Shoulda matchers - biblioteka udostępniana w postaci gemu, rozszerzająca zestaw dostępnych asercji i funkcji pomocniczych,
\item FactoryGirl Rails - biblioteka udostępniana w postaci gemu, pozwalająca na znaczne uproszczenie procesu tworzenia i zapisywania w bazie danych obiektów testowych,
\item Faker - biblioteka udostępniana w postaci gemu, służąca do automatycznego generowania wartości pól tworzonych obiektów testowych, 
\item niezależna testowa baza danych PostgreSQL.
\end{itemize}

\subsection{Testy jednostkowe}
\label{Chapter711}

Testy jednostkowe pozwalają na weryfikację poprawności działania pojedynczych metod określonej klasy, a zatem umożliwiają precyzyjną lokalizację przyczyny wystąpienia ewentualnych błędów wykonania. W systemie LinkedInGrads stanowią podstawowy rodzaj testów i są wykorzystywane głównie do sprawdzania powiązań i reguł walidacji modeli oraz publicznych akcji kontrolerów. Wprowadzenie do projektu testów jednostkowych pozwoliło na wykrywanie drobnych błędów w możliwie krótkim czasie od ich pojawienia się w systemie. 

\subsection{Testy integracyjne}
\label{Chapter712}

Testy integracyjne pozwalają na weryfikację poprawności współdziałania kilku jednostek systemu, takich jak klasy, moduły czy zasoby. Z uwagi na silne zorientowanie tworzonego systemu na dane, stanowią dla niego bardzo istotny rodzaj testów i są wykorzystywane głównie do sprawdzania wyników złożonych zapytań do bazy danych. Wprowadzenie do projektu testów integracyjnych pozwoliło na upewnienie się, że generowane przez system raporty o absolwentach zawierają właściwe informacje.

\subsection{Testy akceptacyjne}
\label{Chapter713}

Testy akceptacyjne pozwalają na walidację systemu, a zatem potwierdzenie jego zgodności z oczekiwaniami klienta. Z uwagi na prostotę interfejsu systemu LinkedInGrads oraz niewielką liczbę kroków głównego scenariusza, a także istotny koszt czasowy konfiguracji odpowiednich narzędzi, zrezygnowano z wprowadzania do projektu automatycznych testów tego rodzaju na rzecz systematycznie przeprowadzanych testów manualnych. W dodatku D przedstawiono scenariusze manualnych testów akceptacyjnych, opracowane na podstawie przypadków użycia przedstawionych w rozdziale 3.

\subsection{Inne metody zapewniania jakości}
\label{Chapter714}

\subsection{Metodyka pracy i model przyrostowy}

W trakcie realizacji projektu zespół programistyczny pracował zgodnie ze zwinną metodyką Scrum. Metodyka ta opiera się na podejściu iteracyjnym, tj. wprowadzeniu podziału procesu rozwijania produktu na krótkie iteracje (tzw. sprinty, przebiegi bądź przyrosty). Po każdej iteracji zespół dostarczał klientowi działającą wersję produktu, wzbogaconą w stosunku do poprzedniej o pewien zakres funkcji niosących wartość biznesową. Podczas spotkania będącego przeglądem sprintu (ang. Sprint Review) klient miał możliwość zapoznania się z uaktualnionym systemem i przekazania zespołowi informacji zwrotnej na jego temat. Prowadzenie prac implementacyjnych zgodnie z modelem przyrostowym oraz utrzymywanie stałego kontaktu z klientem umożliwiło systematyczną walidację systemu i stanowiło istotną metodę zapewniania jakości rozumianej jako zgodność z oczekiwaniami klienta.

\subsection{Standardy kodowania}

Standardy kodowania stanowią zestaw zaleceń dotyczących konwencji nazewniczych i reguł formatowania kodu źródłowego. Dzięki stosowaniu standardów systemy współtworzone przez wielu programistów zachowują czytelność i jednolitość. Służy to przede wszystkim przyszłym administratorom systemu i osobom, które zamierzają go rozwijać. W trakcie prac nad systemem LinkedInGrads zastosowano standardy kodowania zdefiniowane w The Ruby Style Guide [5]. Weryfikacji zgodności z przyjętymi standardami dokonano za pomocą narzędzia do statycznej analizy kodu źródłowego RuboCop.

\subsection{Ciągła integracja}

Ciągła integracja (ang. continuous integration) jest uznaną praktyką programistyczną, zakładającą regularne integrowanie zmian wprowadzanych do kodu programu przez różnych programistów. W celu realizacji tego założenia, w ramach prac nad systemem LinkedInGrads, na serwerze deweloperskim został zainstalowany serwer ciągłej integracji Jenkins, który po każdej wykrytej zmianie w systemie kontroli wersji samoczynnie wywoływał następujące akcje:

\begin{itemize}
\item pobranie najnowszej wersji kodu z repozytorium SVN,
\item pobranie bibliotek potrzebnych do prawidłowego działania systemu,
\item aktualizacja schematu produkcyjnej i testowej bazy danych,
\item scalenie i minifikacja statycznych plików CSS i JS,
\item wykonanie automatycznych testów jednostkowych i integracyjnych,
\item wykonanie automatycznej analizy kodu pod kątem zgodności z przyjętymi standardami,
\item zatrzymanie i ponowne uruchomienie serwera aplikacji.
Wykorzystanie tak skonfigurowanego narzędzia do ciągłej integracji zagwarantowało regularne i automatyczne przeprowadzanie testów oraz umożliwiło monitorowanie bieżącego stanu systemu. Istotnemu zmniejszeniu uległ również koszt czasowy wdrażania kolejnych wersji systemu na serwer produkcyjny.
\end{itemize}