\chapter{Opis implementacji}
\label{Chapter6}

\section{Wstęp}
\label{Chapter61}

Wprowadzenie. Struktura tego rozdziału nie jest z góry określona, gdyż mocno zależy to od specyfiki projektu. Generalnie w poszczególnych podrozdziałach każdy powinien opisać swoją część z takiego technicznego punktu widzenia. Piszecie, jak zrealizowaliście poszczególne wymagania, jak to wygląda ,,pod maską'', oczywiście też trzeba przyjąć jakiś poziom szczegółowości. W bardzo szczególnych przypadkach chyba może się zdarzyć, że trzeba będzie załączyć fragment jakiegoś kodu źródłowego czy konfiguracji -- generalnie ma to być opisane w taki sposób, że jako osoba nieznająca systemu siadam i wiem, jak i co zrobiliście. 

Ten rozdział ma umożliwić innym osobom modyfikację kodu. Warto zatem opisać najbardziej prawdopodobne scenariusze zmian. Np. jesteśmy świadomi, że w naszym systemie może zaistnieć konieczność zmiany wizualnej jakiegoś raportu. Wówczas należałoby wyjaśnić w jaki sposób należy zmodyfikować kod, aby tego typu zmiany wcielić w życie. 

Podsumowując można tutaj umieszczać dwa typy podrozdziałów:
- tłumaczące działania określonych mechanizmów
- opisujące pewne scenariusze zmiany - tego typu rozdział powinien mieć jasno określony scenariusz zmiany (warto także nawiązać do wymagań) oraz rozwiązanie:


\section{Scenariusz zmiany: modyfikacja raportu}
\label{Chapter62}

\subsubsection{Opis problemu}
System X umożliwia generowanie raportów o stanie finansowym (UC1). Istnieje konieczność zmiany wizualnej w układzie graficznym raportu.

\subsubsection{Rozwiązanie}
...

\section{Mechanizm generowania raportu}
\label{Chapter63}

Rozdział o charakterze opisowym prezentujący implementację konkretnego mechanizmu.

\section{Sekcja ...}
\label{Chapter64}

Dalsze opisy.