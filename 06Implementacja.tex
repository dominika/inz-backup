\chapter{Opis implementacji}
\label{Chapter6}

W powodzeniu realizacji systemu informatycznego istotny jest wybór narzędzi i technologii, które wspomogą zespół programistów i zwiększą jakość rozwiązania. W niniejszym rozdziale opisane zostaną narzędzia wykorzystane podczas wykonywania pracy inżynierskiej oraz struktura projektu wraz z podstawowymi plikami konfiguracji.

\section{Narzędzia}

\subsection{RubyMine}

Zintegrowane środowisko programistyczne (IDE) dedykowane dla programistów Ruby i Ruby on Rails, podstawowa aplikacja wspierająca pracę programisty. Zespół podjął decyzję o wyborze narzędzia RubyMine. Oprócz podstawowej funkcjonalności jak formatowanie kodu, refaktoryzacja, nawigacja po projekcie i łatwe debuggowanie, zapewniło stabilność oraz integrację z systemem kontroli wersji SVN oraz z systemem zarządzania bazą danych PostgreSQL.

\subsection{SVN}

Scentralizowany system kontroli wersji. Umożliwia śledzenie zmian w projekcie, co ułatwia pracę jednoczesną pracę wielu członków zespołu. Kolejne wersje systemu mogą być oznaczane, więc gdy zajdzie potrzeba można natychmiastowo przywrócić system do poprzednich wersji.

\subsection{Jenkins}

Podczas procesu implementacji projektu wykorzystano serwer ciągłej integracji Jenkins. Został on zintegrowany w ten sposób by wraz z każdą nową wersją w repozytorium uruchamiane były testy. W przypadku pomyślnego zbudowania aplikacji zmiany są automatycznie wprowadzane na serwer produkcyjny.

\subsection{Nginx oraz Phusion Passenger}

Konfiguracja serwerów HTTP użyta do uruchomienia sytemu LinkedInGrads w środowisku produkcyjnym. Takie połączenie ma dwie zalety:

\begin{itemize}
\item Bezpieczeństwo - serwery jak Apache i Nginx są bardzo dojrzałe. Użycie ich jako serwery pośredniczące zwiększają bezpieczeństwo uruchomionych na nich aplikacji.
\item Wydajność - niektóre serwery lepiej radzą sobie z obsługą treści statycznych.
\end{itemize}

/////////////////
\section{Scenariusz zmiany: modyfikacja raportu}
\label{Chapter62}

\subsubsection{Opis problemu}
System X umożliwia generowanie raportów o stanie finansowym (UC1). Istnieje konieczność zmiany wizualnej w układzie graficznym raportu.

\subsubsection{Rozwiązanie}
...

\section{Mechanizm generowania raportu}
\label{Chapter63}

Rozdział o charakterze opisowym prezentujący implementację konkretnego mechanizmu.

\section{Sekcja ...}
\label{Chapter64}

Dalsze opisy.