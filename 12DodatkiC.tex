\chapter{Instrukcja instalacji}
\label{Chapterc1}

\section{Instalacja wymaganych komponentów}
Instrukcja dotyczy instalacji w systemach z rodziny Unix.
\section{Instalacja i uruchomienie}
\begin{itemize}
\item Menedżera wersji języka ruby RVM (ang. ruby version manager):
\begin{verbatim}
curl -sSL https://get.rvm.io > rvm_install
chmod +x rvm_install
sudo ./rvm_install
\end{verbatim}
	\begin{itemize}
	\item użytkownik, który będzie korzystał z menedżera powinien być w 		grupie `rvm`, aby to sprawdzić:
	\begin{verbatim} 
	groups
	\end{verbatim}
	\item jeżeli użytkownik nie jest w grupie `rvm` to dodaj go do tej 			grupy:
	\begin{verbatim}
	sudo usermod -a -G linkedin rvm
	\end{verbatim}
	\item jeżeli `rvm` nie jest dostępne:
	\begin{verbatim}
	rvm reload
	\end{verbatim}
	\end{itemize}
\item Ruby:
\begin{verbatim}
rvm install 2.0.0-p247
\end{verbatim}
	\begin{itemize}
	\item ruby 2.0.0 powinien być zainstalowany, domyślny i bieżący:
	\begin{verbatim}
	rvm list
	\end{verbatim}
	oczekiwany wynik:
	\begin{verbatim}
	rvm rubies
=* ruby-2.0.0-p247 [ x86_64 ]
=> - current
=* - current && default
* - default
	\end{verbatim}
	\item jeżeli 2.0.0 nie jest bieżący:
	\begin{verbatim}
	rvm use 2.0.0
	\end{verbatim}
	\item jeżeli 2.0.0 nie jest domyślny:
	\begin{verbatim}
	rvm alias create default ruby-2.0.0-p247
	\end{verbatim}
	\end{itemize}
\item Wymagane pakiety:

\begin{verbatim}
sudo apt-get update
sudo apt-get install libreadline-dev rubygems build-essential
  postgresql postgresql-contrib subversion ia32-libs
  libpam0g:i386 libxslt-dev libxml2-dev libpq-dev
  libcurl4-openssl-dev nodejs libpcre3 libpcre3-dev
\end{verbatim}
\item Passenger:
\begin{verbatim}
rvm all do gem install passenger --version 4.0.17
\end{verbatim}
\item Nginx:
\begin{verbatim}
rvm all do passenger-install-nginx-module --auto-download
  --auto --prefix=/opt/nginx
rvmsudo -E /usr/local/rvm/wrappers/ruby-2.0.0-p247/ruby
  /usr/local/rvm/gems/ruby-2.0.0-p247/gems/passenger-4.0.17/bin/
passenger-install-nginx-module --auto-download
  --auto --prefix=/opt/nginx
\end{verbatim}
	\begin{itemize}
	\item w razie błędów:
	\begin{verbatim}
	rvmsudo rvm get stable && rvm reload && rvmsudo rvm repair all
	\end{verbatim}
	\item zmiany w konfiguracji:
		\begin{itemize}
		\item skopiuj plik głównej konfiguracji:
		\begin{verbatim}
		sudo cp docs/nginx.conf /opt/nginx/conf/nginx.conf
		\end{verbatim}
		UWAGA: upewnij się, że ścieżki do wrappera ruby oraz gemu passenger w pliku konfiguracji są odpowiednie tj. takie jak:
		\begin{verbatim}
		passenger-config --root
		\end{verbatim}
				\item utwórz symlink konfiguracji projektu:
		\begin{verbatim}
		sudo mkdir /opt/nginx/conf/sites
		  && sudo ln -s <katalog projektu>/config/nginx.linkedin.conf/
		  opt/nginx/conf/sites/nginx.linkedin.conf
		\end{verbatim}
		UWAGA: upewnij się, że ścieżka root w pliku konfiguracji wskazuje na katalog $<katalog projektu>/public$
		\end{itemize}
	\item Najczęstsze problemy:
		\begin{itemize}
		\item po próbie restartu/uruchomieniu występuje błąd `cannot bind`:
		\begin{verbatim}
		sudo killall nginx
		\end{verbatim}
		\end{itemize}
	\item Kontrola:
		\begin{itemize}
		\item uruchamianie:
		\begin{verbatim}
		sudo /opt/nginx/sbin/nginx
		\end{verbatim}
		\item przeładowanie konfiguracji:
		\begin{verbatim}
		sudo /opt/nginx/sbin/nginx -s reload
		\end{verbatim}
		\end{itemize}
	\item Tor oraz Privoxy:
	\begin{verbatim}
	sudo apt-get install tor privoxy
	\end{verbatim}
		\begin{itemize}
		\item Upewnij się, że plik $/etc/privoxy/config$ zawiera linię:
		\begin{verbatim}
		forward-socks4a / 127.0.0.1:9050 .
		\end{verbatim}
		\item Zrestartuj usługę:
		sudo /etc/init.d/privoxy restart
		\end{itemize}
	\item w `~/.bashrc` umieść:
	\begin{verbatim}
	export RAILS_ENV=production
	\end{verbatim}
	\end{itemize}
\end{itemize}