\chapter{Zebrane doświadczenia}
\label{Chapter8}

Realizacja systemu LinkedInGrads pozwoliła członkom zespołu programistycznego na zdobycie wartościowego doświadczenia zawodowego w pracy na rzecz rzeczywistego klienta. Poniżej przedstawione zostały wnioski zebrane w trakcie prowadzenia prac nad projektem:

\begin{itemize}
\item W początkowej fazie projektu bardzo ważne jest zdefiniowanie głównego użytkownika i klienta oraz nadrzędnego celu działania systemu. Pierwotnie rolę klienta miało pełnić Centrum Praktyk i Karier Politechniki Poznańskiej, późniejsza zmiana pociągnęła za sobą konieczność ponownego zebrania wymagań i aktualizacji istotnej części dokumentacji projektu.
\item Dla powodzenia całości projektu kluczowy jest przemyślany dobór narzędzi. Pożądane jest, aby programiści mieli solidne podstawy teoretyczne, a najlepiej praktyczne doświadczenie w pracy z wybranymi technologiami. Bardzo dobra znajomość języka Ruby przez część zespołu programistycznego oraz stosunkowo niski próg wejścia zapewniły sprawny przebieg prac programistycznych.
\item Realizacja projektu zgodnie z metodyką Scrum i związane z nią podejście iteracyjne zostały dobrze przyjęte przez zespół programistyczny. Szczególnie istotnymi praktykami okazały się:

	\begin{itemize}
	\item przeglądy sprintu (ang. Sprint Review), pozwalające na 	systematyczną walidację systemu i dostosowywanie go do oczekiwań 	klienta,
	\item retrospektywy (ang. Sprint Retrospective), pozwalające na wyciągnięcie wniosków z dotychczasowych doświadczeń i ciągłe ulepszanie metod pracy,
	\item codzienne spotkania (ang. Daily Scrum), pozwalające na zidentyfikowanie bieżących wyzwań i problemów oraz wspierające przepływ informacji w zespole.
	\end{itemize}

\item Rozmowy i spotkania związane z realizacją projektu zazwyczaj trwały dłużej niż planowano. Warto przed każdym spotkaniem narzucić zespołowi górne ograniczenie czasu i egzekwować jego przestrzeganie. Dyscypliną i stanowczością w tym zakresie powinien się wykazać prowadzący spotkanie  "Mistrz Młyna" (ang. Scrum Master).
\item Korzystanie z narzędzi do ciągłej integracji umożliwia automatyzację wykonywania testów i znacznie skraca czas wdrożenia. Wymienione korzyści rosną wraz z rozmiarem systemu i czasem trwania projektu. W przypadku systemu LinkedInGrads narzędzia do ciągłej integracji nie zostały niestety skonfigurowane na początku projektu. W konsekwencji jeden ze sprintów został zakończony z opóźnieniem spowodowanym problemami z wdrożeniem systemu na serwer. Po wprowadzeniu systemu Jenkins problemy już się nie pojawiły.
\item Tworzenie i utrzymywanie obszernej dokumentacji projektowej jest utrudnione w projektach, w których dynamicznie zmieniają się wymagania. Prowadzi to do sytuacji, w których dokumenty prezentują stan niezgodny z rzeczywistością, a ich aktualizacja nie jest przeprowadzania z uwagi na zbyt duże obciążenie czasowe i niski priorytet. W przypadku pracy w niewielkim, lokalnym zespole nad ograniczonym czasowo projektem słuszne okazało się założenie metodyk zwinnych o wyższości działającego oprogramowanie nad formalną dokumentacją.

\end{itemize}